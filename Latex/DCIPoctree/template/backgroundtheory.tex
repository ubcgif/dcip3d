% All the theory behind the program
\section{Background theory}

\subsection{Introduction}

This manual presents theoretical background, numerical examples, and explanation for implementing the program library \prog. This suite of algorithms, developed at the UBC-Geophysical Inversion Facility, is needed to invert magnetic responses over a three-dimensional susceptibility distribution. The manual is designed so that a geophysicist who is familiar with the magnetic experiment, but who is not necessarily versed in the details of inverse theory, can use the codes and invert his or her data.

A magnetic experiment involves measuring the anomalous magnetic field produced by magnetically susceptible materials beneath the surface, which have been magnetized by the earth's main magnetic field. The material with susceptibility $\kappa(x,y,z)$ is magnetized when the earth's main field with flux intensity $\mathbf{B}_o$ impinges upon the subsurface formation. The magnetized material gives rise to a
magnetic field, $\mathbf{B}_a$, which is superimposed on the inducing field to produce a total, or resultant, field. By measuring the resultant field and removing the inducing field from the measurements through numerical processing, one obtains the distribution of the anomalous field due to the susceptible material. Very often, the susceptible materials underground possess a certain amount of natural remanent magnetization. In this program library, however, we make the assumption that no remanent magnetization is present and restrict our attention to induced magnetization.

The data from a typical magnetic survey is a set of magnetic field measurements acquired over a 2D grid above the surface or along a number of boreholes within the volume of interest. These data are first processed to yield an estimate of the anomalous field due to the susceptible material in the area. The goal of the magnetic inversion is to obtain, from the extracted anomaly data, quantitative information about the distribution of the magnetic susceptibility in the ground. Thus it is assumed that the input data to the inversion program is the extracted residual anomaly and the programs in the library are developed accordingly.

\subsection{Forward modelling}
\subsubsection{General formulation}

For a given inducing field $\mathbf{B}_o$, the magnetization $\mathbf{J}$ depends upon the susceptibility through a differential equation. However, to the first order approximation when the actual susceptibility is very small, as is most often the case with material encountered in mineral explorations, the magnetization  is proportional to the susceptibility and is given by the product of susceptibility with inducing magnetic field $\mathbf{H}_o$,

\begin{equation}
\label{eq:magnetization}
\mathbf{J}=\bm{\kappa}\mathbf{H}_o,
\end{equation}

where $\mathbf{H}_o=\mathbf{B}_o / \mu_o$ and $\mu_o$ is the free-space magnetic permeability. This essentially ignores the self-demagnetization effect by which the secondary field reduces the total inducing field within the susceptible region and results in a weaker magnetization than that given by equation \ref{eq:magnetization}.

The anomalous field produced by the distribution of magnetization $\mathbf{J}$ given by the following integral equation with a dyadic Green's function:

\begin{equation}
\label{eq:greensf}
\mathbf{B}_a(\mathbf{r})=\frac{\mu_0}{4\pi}\int\limits_V \nabla \nabla \frac{1}{\left |\mathbf{r}-\mathbf{r}_o\right |}\cdot\mathbf{J} dv,
\end{equation}

where $\mathbf{r}$ is the position of the observation point and $V$ represents the volume of magnetization. The above equation is valid for observation locations above the earth's surface. It is also valid in the boreholes provided we assume that the magnetic permeability is $\mu_o$.

When the susceptibility is constant within a volume of source region, the above equation can be written in matrix form as:

\begin{equation}
\label{eq:bmatrix}
\mathbf{B}_a=\mu_o\left( \begin{array}{ccc}
T_{11} & T_{12} & T_{13} \\
T_{21} & T_{22} & T_{23}\\
T_{31} & T_{32} & T_{33}
\end{array} \right)\kappa\mathbf{H}_o\equiv \mu_o\kappa\mathbf{T}\mathbf{H}_o.
\end{equation}
%
The tensor $\mathbf{T}_{ij}$ is given by
%
\begin{equation}
\label{eq:tij}
\mathbf{T}_{ij}=\frac{1}{4\pi}\int\limits_V\frac{\partial}{\partial x_i}\frac{\partial}{\partial x_j}\frac{1}{\left |\mathbf{r}-\mathbf{r}_o\right |}dv, \mbox{  for }i=1,3 ; j=1,3,
\end{equation}
%
and where $x_1$, $x_2$, and $x_3$ represent $x-, y-$, and $z-$directions, respectively. The expressions of $\mathbf{T}_{ij}$  for a cuboidal source volume can be found in \cite{Bhattacharyya64} and \cite{Sharma66} (here we assume that the effect of borehole cavity can be neglected). Since $\mathbf{T}$ is symmetric and its trace is equal to $-1$ when the observation is inside the cell and is $0$ when the observation is outside the cell, only five independent elements need to be calculated.

Once $\mathbf{T}$ is formed, the magnetic anomaly $\mathbf{B}_a$ and its projection onto any direction of measurement are easily obtained by the inner product with the directional vector. The projection of the field $\mathbf{B}_a$ onto different directions yields different anomalies commonly obtained in the magnetic survey. For instance, the vertical anomaly is simply $B_{az}$, the vertical component of $\mathbf{B}_a$, whereas the total field anomaly is, to first order, the projection of $\mathbf{B}_a$ onto the direction of the inducing field $\mathbf{B}_o$.

\subsubsection{Borehole data}

In a borehole experiment, the three components are measured in the directions of local coordinate axes ($l_1$, $l_2$, $l_3$) defined according to the borehole orientation. Assuming that the borehole dip $\theta$ is measured downward from the horizontal surface and azimuth $\varphi$ is measured eastward from the North; a commonly used convention has the $l_3$-axis pointing downward along borehole, $l_1$-axis pointing perpendicular to the borehole in the direction of the azimuth. The $l_2$-axis
completes the right-handed coordinate system and is $90^\circ$ clockwise from the azimuth and perpendicular to the borehole. Based upon the above definition the rotation matrix that transforms three components of a vector in the global coordinate system to the components in the local coordinates is given by
\begin{equation}
\label{eq:rotation}
\mathbf{R}=\left( \begin{array}{ccc}
\cos\varphi \sin\theta & \sin\varphi \sin\theta & -\cos\theta \\
-\sin\varphi & \cos\varphi & 0\\
\cos\varphi \cos\theta & \sin\varphi \cos\theta & \sin\theta
\end{array} \right)
\end{equation}
If a vector is defined in local coordinates as $(l_1,l_2,l_3)^T$, and in global coordinates as $(g_1,g_2,g_3)^T$, then the following two relations hold:
\begin{eqnarray}
\label{eq:vectors}
(l_1,l_2,l_3)^T = \mathbf{R}(g_1,g_2,g_3)^T \nonumber \\
(g_1,g_2,g_3)^T = \mathbf{R}^T(l_1,l_2,l_3)^T
\end{eqnarray}
%
The rotation matrix $\mathbf{R}$ therefore allows measured components in local coordinates to be rotated into global coordinate, or the components of the regional field to be rotated into local coordinates for use in regional removal.

\subsubsection{Numerical implementation of forward modelling}

We divide the region of interest into a set of 3D cuboidal cells by using a 3D orthogonal mesh and assume a constant susceptibility within each cell. By equation \ref{eq:magnetization}, we have a uniform magnetization within each cell and its field anomaly can be calculated using equations \ref{eq:bmatrix} and \ref{eq:vectors}. The actual anomaly that would be measured at an observation point is the sum of field produced by all cells having a non-zero susceptibility value. The calculation involves the evaluation of equation \ref{eq:bmatrix} in a 3D rectangular domain define by each cell. The program that performs this calculation is \codeName{MAGFOR3D}. As input parameters, the coordinates of the observation points and the inclination and declination of the anomaly direction must be specified for each datum. For generality, each component in a multi-component data set is specified as a separate datum with its own location and direction of projection.

\subsection{Inversion methodology}
\label{inversionmethodology}
Let the set of extracted anomaly data be $\mathbf{d} = (d_1,d_2,...,d_N)^T$ and the susceptibility of cells in the model be $\bm{\kappa} = (\kappa_1,\kappa_2,...,\kappa_M)^T$. The two are related by the sensitivity matrix

\begin{equation}
\label{eq:sens}
\mathbf{d}=\mathbf{G}\bm{\kappa}.
\end{equation}
%
The matrix has elements $g_{ij}$ which quantify the contribution to the $i^{th}$ datum due to a unit susceptibility in the $j^{th}$ cell. The program \codeName{MAGSEN3D} performs the calculation of the sensitivity matrix, which is to be used by the subsequent inversion. The sensitivity matrix provides the forward mapping from the model to the data during the entire inverse process. We will discuss its efficient representation via the wavelet transform in a separate section.

The inverse problem is formulated as an optimization problem where an objective function of the model is minimized subject to the constraints in equation \ref{eq:sens}. For magnetic inversion the first question that arises concerns definition of the ``model''. We choose magnetic susceptibility $\kappa$ as the model since the anomalous field is directly proportional to the susceptibility. This is the choice for the inversion program \codeName{MAGINV3D}. For generality, we introduce the generic symbol $m$ for the model element. Having define a ``model'' we next construct an objective function which, when minimized, produces a model that is geophysically interpretable. The details of the objective function are problem dependent but generally we need the flexibility to be close to a reference model $m_o$ and also require that the model be relatively smooth in three spatial directions. Here we adopt a right handed Cartesian coordinate system with positive north and positive down. Let the model objective function be

\begin{eqnarray}
\label{eq:mof}
\phi_m(m) &=& \alpha_s\int\limits_V w_s\left\{w(\mathbf{r})[m(\mathbf{r})-{m}_o] \right\}^2dv + \alpha_x\int\limits_V w_x \left\{\frac{\partial w(\mathbf{r})[m(\mathbf{r})-{m}_o]}{\partial x}\right\}^2dv \\ \nonumber
&+& \alpha_y\int\limits_V w_y\left\{\frac{\partial w(\mathbf{r})[m(\mathbf{r})-{m}_o]}{\partial y}\right\}^2dv +\alpha_z\int\limits_V\ w_z\left\{\frac{\partial w(\mathbf{r})[m(\mathbf{r})-{m}_o]}{\partial z}\right\}^2dv,
\end{eqnarray}
%
where the functions $w_s$, $w_x$, $w_y$ and $w_z$ are spatially dependent, while $\alpha_s$, $\alpha_x$, $\alpha_y$ and $\alpha_z$ are coefficients, which affect the relative importance of different components in the objective function. Here the function $w(\mathbf{r})$ is a generalized depth weighting function. The purpose of this function is to counteract the geometrical decay of the sensitivity with the distance from the observation location so that the recovered susceptibility is not concentrated near the observation locations. The details of the depth weighting function will be discussed in the next section.

The objective function in equation \ref{eq:mof} has the flexibility to construct many different models. The reference model $m_o$ may be a general background model that is estimated from previous investigations or it could be a zero model. The reference model would generally be included in the first component of the objective function but it can be removed if desired from the remaining terms; often we are more confident in specifying the value of the model at a particular point than in supplying an estimate of the gradient. The choice of whether or not to include $m_o$ in the derivative terms can have significant effect on the recovered model. The relative closeness of the final model to the reference model at any location is controlled by the function $w_s$. For example, if the interpreter has high confidence in the reference model at a particular region, he can specify $w_s$ to have increased amplitude there compared to other regions of the model. The weighting functions $w_x$, $w_y$, and $w_z$ can be designed to enhance or attenuate structures in various regions in the model domain. If geology suggests a rapid transition zone in the model, then a decreased weighting for flatness can be put there and the constructed model will exhibit higher gradients provided that this feature does not contradict the data.

To perform a numerical solution, we discretize the model objective function in equation \ref{eq:mof} using a finite difference approximation on the mesh defining the susceptibility model. This yields:

\begin{eqnarray}
\label{eq:modobjdiscr}
\phi_m(\mathbf{m}) = (\mathbf{m}-\mathbf{m}_o)^T(\alpha_s \mathbf{W}_s^T\mathbf{W}_s+\alpha_x \mathbf{W}_x^T\mathbf{W}_x+\alpha_y \mathbf{W}_y^T\mathbf{W}_y+\alpha_z \mathbf{W}_z^T\mathbf{W}_z)(\mathbf{m}-\mathbf{m}_o), \nonumber\\
\equiv(\mathbf{m}-\mathbf{m}_o)^T(\mathbf{W}_m^T\mathbf{W}_m)(\mathbf{m}-\mathbf{m}_o), \nonumber\\
=\left \| \mathbf{W}_m(\mathbf{m}-\mathbf{m}_o) \right \|^2,
\end{eqnarray}
where $\mathbf{m}$ and $\mathbf{m}_o$ are $M$-length vectors. The individual matrices $\mathbf{W}_s$ , $\mathbf{W}_x$, $\mathbf{W}_y$, and $\mathbf{W}_z$ are straight-forwardly calculated once the model mesh and the weighting functions $w(\mathbf{r})$ and $w_s$ , $w_x$, $w_y$, $w_z$ are defined. The cumulative matrix $\mathbf{W}_m^T\mathbf{W}_m$ is then formed.

The next step in setting up the inversion is to define a misfit measure. Here we use the $l_2$-norm measure
\begin{eqnarray}
\label{eq:phid}
\phi_d = \left\| \mathbf{W}_d(\mathbf{G}\mathbf{m}-\mathbf{d})\right\|^2.
\end{eqnarray}
%
For the work here we assume that the contaminating noise on the data is independent and Gaussian with zero mean. Specifying $\mathbf{W}_d$ to be a diagonal matrix whose $i^{th}$ element is $1/\sigma_i$, where $\sigma_i$ is the standard deviation of the $i^{th}$ datum, makes $\phi_d$ a chi-squared variable distributed with $N$ degrees of freedom. Accordingly $E[\chi^2]=N$ provides a target misfit for the inversion.

The inverse problem is solved by finding a model m which minimizes $\phi_m$ and misfits the data by a pre-determined amount. Since the susceptibility is positive by definition we also need to impose the constraint that all model elements be positive. More generally, we can require that the recovered model ($m$) lies between imposed lower ($m^l$) and upper ($m^u$) bounds. Thus the solution is obtained by the following minimization problem of a global objective function $\phi$,

\begin{eqnarray}
\label{eq:globphi}
\min \phi = \phi_d+\beta\phi_m \\
\mbox{s. t. } m^l\leq m\leq m^u, \nonumber
\end{eqnarray}
%
where $\beta$ is a trade off parameter that controls the relative importance of the model norm and data misfit. When the standard deviations of data errors are known, the acceptable misfit is given by the expected value $\phi_d$ and we will search for the value of $\beta$ via an L-curve criterion \cite[]{Hansen00} that produces the expected misfit. Otherwise, an estimated value of $\beta$ can be obtained using a generalized cross validation (GCV) estimate \cite[]{GolubHeath79} or by providing a user-defined $\beta$.

To solve the optimization problem when constraints are imposed we use the projected gradients method \cite[]{CalamaiMore87,Vogel02}. Previous versions of \programName have used the logarithmic barrier method \cite[]{Wright97,NocedalWright99}, however the projected gradient method is considerably faster.

This methodology provides a basic framework for solving 3D magnetic inversion with arbitrary observation locations. The basic components are the forward modelling, a model objective function that incorporates a depth weighting and information about the reference model, a data misfit function, a trade off parameter that ultimately determines how well the data will be fit and an optimization algorithm that minimizes an objective function, subject to bound constraints.

The weighting function is directly incorporated in the sensitivity file generated by program \codeName{MAGSEN3D}. This program allows user to specify whether to use the depth weighting or the distance weighting for surface data. When borehole data are present, however, distance weighting must be used.

\subsection{Depth Weighting and Distance Weighting}

It is a well-known fact that static magnetic data have no inherent depth resolution. A numerical consequence of this is that when an inversion is performed, which minimizes $\int m(\mathbf{r})^2 dv$, subject to fitting the data, the constructed susceptibility is concentrated close to the observation locations. This is a direct manifestation of the kernel's decay with the distance between the cell and observation locations. Because of the rapidly diminishing amplitude, the kernels of magnetic data are not sufficient to generate a function, which possess significant structure at locations that are far away from observations. In order to overcome this, the inversion needs to introduce a weighting to counteract this natural decay. Intuitively, such a weighting will approximately cancel the decay and give cells at different locations equal probability to enter into the solution with a non-zero susceptibility.

% Subsections on depth and distance weighting
\input depthdistweight.tex

\subsection{Wavelet Compression of Sensitivity Matrix}
\label{wavelet}

The two major obstacles to the solution of large scale magnetic inversion problem are the large amount of memory required for storing the sensitivity matrix and the CPU time required for the application of the sensitivity matrix to model vectors. The \prog~ program library overcomes these difficulties by forming a sparse representation of the sensitivity matrix using a wavelet transform based on compactly supported, orthonormal wavelets. For more details, the users are referred to \cite{LiOldenburg03,LiOldenburg10}. In the following, we give a brief description of the method necessary for the use of the \prog~ library.

Each row of the sensitivity matrix in a 3D magnetic inversion can be treated as a 3D image and a 3D wavelet transform can be applied to it. By the properties of the wavelet transform, most transform coefficients are nearly or identically zero. When coefficients of small magnitudes are discarded (the process of thresholding), the remaining coefficients still contain much of the necessary information to reconstruct the sensitivity accurately. These retained coefficients form a sparse representation of the sensitivity in the wavelet domain. The need to store only these large coefficients means that the memory requirement is reduced. Further, the multiplication of the sensitivity with a vector can be carried out by a sparse multiplication in the wavelet domain. This greatly reduces the CPU time. Since the matrix-vector multiplication constitutes the core computation of the inversion, the CPU time for the inverse solution is reduced accordingly. The use of this approach increases the size of solvable problems by nearly two orders of magnitude.

Let $\mathbf{G}$ be the sensitivity matrix and $\mathcal{W}$ be the symbolic matrix-representation of the 3D wavelet transform. Then applying the transform to each row of $\mathbf{G}$ and forming a new matrix consisting of rows of transformed sensitivity is equivalent to the following operation:

\begin{equation}
\label{eq:senswvt}
\widetilde{\mathbf{G}}=\mathbf{G}\mathcal{W}^T,
\end{equation}
%
where $\widetilde{\mathbf{G}}$ is called the transformed matrix. The thresholding is applied to individual rows of $\mathbf{G}$ by the following rule to form the sparse representation $\widetilde{\mathbf{G}}^S$,

\begin{equation}
\label{eq:elemg}
\widetilde{g}_{ij}^{s}=\begin{cases}
\widetilde{g}_{ij} & \mbox{if } \left|\widetilde{g}_{ij}\right| \geq \delta _i \\
0 & \mbox{if } \left|\widetilde{g}_{ij}\right| < \delta _i
\end{cases}, ~~ i=1,\ldots,N,
\end{equation}
%
where $\delta _i$ is the threshold level, and $\widetilde{g}_{ij}$ and $\widetilde{g}_{ij}^{s}$ are the elements of $\widetilde{\mathbf{G}}$ and	 $\widetilde{\mathbf{G}}^S$, respectively. The threshold level $\delta _i$ are determined according to the allowable error of the reconstructed sensitivity, which is measured by the ratio of norm of the error in each row to the norm of that row, $r_i(\delta_i)$. It can be evaluated directly in the wavelet domain by the following expression:

\begin{equation}
\label{eq:rhoi}
r_i(\delta_i)=\sqrt{\frac{\underset{\left | {\widetilde{g}_{ij}} \right |<\delta_i}\sum{\widetilde{g}_{ij}}^2}{\underset{j}\sum{\widetilde{g}_{ij}^2}}}, ~~i=1,\ldots,N,
\end{equation}
%
Here the numerator is the norm of the discarded coefficients. For each row we choose $\delta_i$ such that $r_i(\delta_i)=r^*$, where $r^*$ is the prescribed reconstruction accuracy. However, this is a costly process. Instead, we choose a representative row, $i_o$, and calculate the threshold level $\delta_{i_o}$. This threshold is then used to define a relative threshold $\epsilon =\delta_{i_{o}}/ \underset{j}{\max}\left | {\widetilde{g}_{ij}} \right |$. The absolute threshold level for each row is obtained by

\begin{equation}
\label{eq:deltai}
\delta_i = \epsilon \underset{j}{\max}\left | {\widetilde{g}_{ij}} \right|, ~~i=1,\ldots,N.
\end{equation}

The program that implements this compression procedure is \codeName{MAGSEN3D}. The user is asked to specify the relative error $r^*$ and the program will determine the relative threshold level $\delta_i$. Usually a value of a few percent is appropriate for $r^*$. When both surface and borehole data are present, two different relative threshold levels are calculated by choosing a representative row for surface data and another for borehole data. For experienced users, the program also allows the direct input of the relative threshold level.